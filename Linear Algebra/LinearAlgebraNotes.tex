\documentclass{article}
\usepackage[utf8]{inputenc}
\usepackage{amsmath,tikz,tcolorbox}

\title{Linear Algebra Notes}
\author{sumit singh}
\date{\today}

\begin{document}

\maketitle

\section{The Equations}

\begin{equation}
    \min_{\forall i} \sqrt{wt_1 (GoalDT - DT_i)^2 + wt_2 (Goal OT - OT_i)^2 +wt_3 (Goal CR - CR_i)^2}
\end{equation}
\subsection{Cross Product in the form of unit vector}
\begin{equation}
    \Vec{a} \times \Vec{b} = |a| |b| \sin\theta   \hat{n}
\end{equation}
\subsection{Cross Product in the form of a Matrix}
\begin{equation}
    \vec{a} = \begin{pmatrix}
        a_1 \\ a_2 \\ a_3
    \end{pmatrix}
\end{equation}

\begin{equation}
    \vec{b} = \begin{pmatrix}
        b_1 \\ b_2 \\ b_3
    \end{pmatrix}
\end{equation}



\begin{equation}
   \vec{a} \times \vec{b} = (a_2 b_3 -a_3 b_2)i  - (a_1b_3 - a_3 b_1)j + (a_1 b_2 - a_2 b_1)k
\end{equation}

\begin{equation}
    \vec{a} \times \vec{b} = \begin{vmatrix}
        i & j & k \\
        a_1 & a_2 & a_3 \\
        b_1 & b_2 & b_3 \\
    \end{vmatrix} = (a_2b_3-a_3b_2)i - (a_1b_3 -a_3b_1)j + (a_1b_2-a_2 b_1)k
\end{equation}


\begin{itemize}
    \item Scalar or Dot Product
    \item Matrix or diadic product
    \item Vector or cross product
\end{itemize}
$\boldsymbol{x} \& \boldsymbol{z}$  are column vectors. :
\begin{equation}
    \boldsymbol{x} = \begin{bmatrix}
        x_0 & x_1 & x_2
    \end{bmatrix}^{T}
\end{equation}

\begin{equation}
    \boldsymbol{z} = \begin{bmatrix}
        z_0 & z_1 & z_2 
    \end{bmatrix}^T
\end{equation}

\begin{flalign}
     &  &  \boldsymbol{x}^T \cdot \boldsymbol{z}= \boldsymbol{x} \bullet  \boldsymbol{z}  & = x_0 z_0. + x_1 z_1 + x_2 z_2 &  & \text{Scalar or Dot Product}\label{eq:a’} \\
     &  &  \boldsymbol{x}^T \cdot \boldsymbol{z}= \boldsymbol{x} \circ  \boldsymbol{z}  & = =\begin{bmatrix}
        x_0z_0 & x_0z_1 & x_0z_2 \\
        x_1z_0 & x_1z_1 & x_1z_2 \\
        x_2z_0 & x_2z_1 & x_2z_2 \\ 
    \end{bmatrix} &  & \text{Matrix or Diadic Product}\label{eq:c’} \\
     &  & \boldsymbol{x} \times \boldsymbol{z}  & = \begin{bmatrix}
        x_1z_2 - x_2z1  \\
        -x_0z_2 + x_2z_0 \\
        x_0z_1 - x_1z_0  \\ 
    \end{bmatrix}  &  &  \text{Vector or Cross Product}\label{eq:b’}
    \end{flalign}

\begin{tcolorbox}[title=Vector or Cross Product,colback=red!30!yellow,colframe=blue!80!white]
    
    \begin{equation}
        \boldsymbol{x} \times \boldsymbol{z}  = \begin{bmatrix}
        x_1z_2 - x_2z1  \\
        -x_0z_2 + x_2z_0 \\
        x_0z_1 - x_1z_0  \\ 
    \end{bmatrix}
    \end{equation}
\end{tcolorbox}
\section{Introduction}
Partition the input space into recursively generated rectangles. Recursively splitting of region. \\
Biggest advantage of DT is interpretability. (At the other extreme is neural networks which is incomprehensible) . This helps in very compactly describing  the segmentation as a tree. An easily understandable tree. \\
Neural networks are universal approximators. So are DTs. The parameters can grow unbounded. \\
In DT, you can repeat the variables. \\
DTs are \textbf{non parametric}. It can keep growing. You can keep adding the parameters as you go along. \\
\section{Regression Trees}
So far, we did partitioning. Now regression trees.\\
\section{References}
\end{document}